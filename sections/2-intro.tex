%-------------------------------------------------------------------------------
% 2-intro.tex 
% 
% Файл главы с введением
%
% Автор шаблона: Гордеев Иван
%-------------------------------------------------------------------------------

\section*{\center Введение}
\addcontentsline{toc}{section}{Введение}

Здесь и далее приведено \todo{примерное содержание НИР}. Показана примерная структура НИР и использование основных \LaTeX команд для генерации рисунков, таблиц, формул и т.\,д. 

Рентгеновская установка предназначена для облучения клеточных культур в соответствии с задаваемыми значениями дозы и/или времени облучения. Проводится моделирование различных режимов работы установки с последующим сравнением полученных результатов с результатами эксперимента, проведенного с использованием радиохромных пленок.

Альтернативой прямым экспериментальным измерениям выступает моделирование методом Монте-Карло (МК), которое позволяет упростить процесс получения необходимых параметров без существенных потерь в точности. К МК программам относятся: MCNP~\cite{MCNPweb}, FLUKA~\cites{Ferrari2005,Boehlen2014}, PHITS~\cite{Sato2018}, GEANT4~\cite{Agostinelli2003}. 

В данной работе при помощи Монте-Карло кода FLUKA определяются значения поглощенной дозы и кермы, а также исследуется явление электронного равновесия. 

